\documentclass{article}
\usepackage[margin=0.75in]{geometry}
\usepackage{fancyhdr}
\pagestyle{fancy}
\lhead{}
\chead{}
\rhead{QPE Circuit}
\lfoot{}
\cfoot{\thepage}
\rfoot{}
\usepackage{amsmath, amsthm, amssymb}
\usepackage{mathtools}
\usepackage{bbold} % allow an identity elem 1
\usepackage{graphicx}
\usepackage{caption}
\usepackage[hidelinks]{hyperref}
\usepackage{changepage}
\usepackage{lipsum}
\usepackage{scrextend} % for block indents (for example)
\usepackage{color, colortbl} % For Coloring Table Elements
\usepackage{listings}
\usepackage{makecell}  % allows more complex table formatting
\usepackage{multirow}
\usepackage{proof}     % for inference trees
\usepackage{subcaption}
\usepackage{enumitem}  % allows customization of enumerate

\usepackage{wrapfig}   % allows wrapping text around figures
\usepackage{sidecap}   % allows side captions on figures
\sidecaptionvpos{figure}{t}
\usepackage{physics}   % provides abs{} and norm{}, as well as bra{} and ket{} notations

% for the makecell package:
\renewcommand\theadalign{bc}
\renewcommand\theadfont{\bfseries}
\renewcommand\theadgape{\Gape[4pt]}
\renewcommand\cellgape{\Gape[4pt]}

% customizing enumeration environment
\renewcommand{\labelenumi}{\textbf{(\arabic{enumi})}}
\renewcommand{\labelenumii}{\textbf{(\textit{\alph{enumii}})}}

\addtolength{\hoffset}{-1cm}
\addtolength{\textwidth}{1cm}
\graphicspath{ {IMAGES/} }

\definecolor{Gray}{gray}{0.9}
\newcolumntype{g}{>{\columncolor{Gray}}c}

% for bibliography and citations
\usepackage{biblatex}
\addbibresource{bibliography.bib}

% for script lettering, e.g.
% for Fourier transform
\usepackage[mathscr]{euscript}

% For quantum circuit diagrams
% \usepackage{qcircuit}
\usepackage{tikz}
\usetikzlibrary{quantikz}

\title{QPE Circuit Using \texttt{quantikz}}
\author{Warren D. (Hoss) Craft}
\date{Fall 2023}

\begin{document}
\maketitle
%\tableofcontents

\flushleft

Two constructions of the QPE circuit using the \texttt{quantikz} package. The first and more detailed one we used in the Prove-It-related paper accepted for publication in \textit{Physical Review A} (2023). The more condensed version we used in a poster presentation for SQuInT (October 2023) in Albuquerque, NM.

%%%%%%%%%%%%%%%%%%%%%%%
% Circuit Diagram:    %
% Full QPE Circuit    %
%%%%%%%%%%%%%%%%%%%%%%%

\begin{figure*}[htb]
    \captionsetup{font=small}
    \centerline{
    \resizebox{0.9\textwidth}{!}{
    \begin{quantikz}
    % ROW 1
    \lstick[wires=6]{1st Register:\\$t$ qubits} &
    \lstick{$\ket{0}$} & \gate{H}\slice{1} &
        \ctrl{6} & \qw & \ \ldots\ \qw & \qw & \qw\slice{2} & \gate[wires=6, nwires={4}]{\text{QFT}^{\dagger}}\slice{3} & \meter{} & \rstick[wires=6]{$\ket{2^t \tilde{\varphi}}$}\qw\\
    % ROW 2
    &\lstick{$\ket{0}$} & \gate{H} &
        \qw & \ctrl{5} & \ \ldots\ \qw & \qw & \qw & \qw & \meter{} & \qw\\
    % ROW 3
    &\lstick{$\ket{0}$} & \gate{H} &
        \qw & \qw & \ \ldots\ \qw & \qw & \qw & \qw & \meter{} & \qw\\
    % ROW 4
    &\vdots & \vdots &
        \vdots & \vdots & \vdots & \vdots & \vdots & \vdots & \vdots\\
    % ROW 5
    &\lstick{$\ket{0}$} & \gate{H} &
        \qw & \qw & \ \ldots\ \qw & \ctrl{2} & \qw & \qw & \meter{} & \qw\\
    % ROW 6
    &\lstick{$\ket{0}$} & \gate{H} &
        \qw & \qw & \ \ldots\ \qw & \qw & \ctrl{1} & \qw & \meter{} & \qw\\
    % ROW 6
    \lstick[wires=1]{2nd Register:\\$s$ qubits }
    &\lstick{$\ket{u}$} & \qwbundle[alternate]{} &
        \gate{U^{2^{t-1}}}\qwbundle[alternate]{} & \gate{U^{2^{t-2}}}\qwbundle[alternate]{} & \ \ldots\ \qwbundle[alternate]{} &
        \gate{U^{2^{1}}}\qwbundle[alternate]{} &
        \gate{U^{2^{0}}}\qwbundle[alternate]{} & \qwbundle[alternate]{} & \qwbundle[alternate]{} & \rstick{$\ket{u}$}\qwbundle[alternate]{}
    \end{quantikz}
    }
    }
    \caption{Quantum circuit implementing the QPE algorithm. From initial first-register state $\ket{0}_{t}$ and second-register state $\ket{u}$, we apply Hadamards to the first register lines, controlled $U^{2^j}$ to the second register, and an inverse quantum Fourier transform $\text{QFT}^{\dagger}$ to the first register, leading to first-register measurement $\ket{2^t \tilde{\varphi}}$, where $\tilde{\varphi}$ is an estimate of the phase.}
    \label{fig:QPE_circuit}
\end{figure*}

%%%%%%%%%%%%%%%%%%%%%%%%%%
% Circuit Diagram:       %
% Condensed QPE Circuit  %
%%%%%%%%%%%%%%%%%%%%%%%%%%

\begin{figure*}
    \captionsetup{font=small}
    \centerline{
    % \resizebox{0.9\textwidth}{!}
    {
    \begin{quantikz}[column sep=0.5cm, row sep=0.2cm]
    % ROW 1
    \lstick[wires=4]{1st Register:\\$t$ qubits} &
    \lstick{$\ket{0}$} & \gate{H}\slice{1} &
        \ctrl{4} & \qw & \ \ldots\ \qw & \qw\slice{2} & \gate[wires=4, nwires={3}]{\text{QFT}^{\dagger}}\slice{3} & \meter{} & \rstick[wires=4]{$\ket{2^t \tilde{\varphi}}$}\qw\\
    % ROW 2
    &\lstick{$\ket{0}$} & \gate{H} &
        \qw & \ctrl{3} & \ \ldots\ \qw & \qw & \qw & \meter{} & \qw\\
    % ROW 3
    % &\lstick{$\ket{0}$} & \gate{H} &
    %     \qw & \qw & \ \ldots\ \qw & \qw & \qw & \qw & \meter{} & \qw\\
    % ROW 4
    &\vdots & \vdots &
        \vdots & \vdots & \ddots & \vdots & \vdots & \vdots\\
    % ROW 5
    % &\lstick{$\ket{0}$} & \gate{H} &
    %     \qw & \qw & \ \ldots\ \qw & \ctrl{2} & \qw & \qw & \meter{} & \qw\\
    % ROW 6
    &\lstick{$\ket{0}$} & \gate{H} &
        \qw & \qw & \ \ldots\ \qw & \ctrl{1} & \qw & \meter{} & \qw\\
    % ROW 6
    \lstick[wires=1]{2nd Register:\\$s$ qubits }
    &\lstick{$\ket{u}$} & \qwbundle[alternate]{} &
        \gate{U^{2^{t-1}}}\qwbundle[alternate]{} & \gate{U^{2^{t-2}}}\qwbundle[alternate]{} & \ \ldots\ \qwbundle[alternate]{} &
        \gate{U^{2^{0}}}\qwbundle[alternate]{} & \qwbundle[alternate]{} & \qwbundle[alternate]{} & \rstick{$\ket{u}$}\qwbundle[alternate]{}
    \end{quantikz}
    }
    }
    \caption{A slightly condensed version of the circuit from Fig. \ref{fig:QPE_circuit} implementing the QPE algorithm. From initial first-register state $\ket{0}_{t}$ and second-register state $\ket{u}$, we apply Hadamards to the first register lines, controlled $U^{2^j}$ to the second register, and an inverse quantum Fourier transform $\text{QFT}^{\dagger}$ to the first register, leading to first-register measurement $\ket{2^t \tilde{\varphi}}$, where $\tilde{\varphi}$ is an estimate of the phase.}
    \label{fig:QPE_circuit_condensed}
\end{figure*}



% \clearpage
% ========== %
% END MATTER %
% ========== %
% \newpage
\printbibliography

\end{document}